\documentclass{exam}
\usepackage[utf8]{inputenc}
\title{Homework 1}
\author{Vincent Yang}

\begin{document}
\pagestyle{headandfoot}
\runningheadrule
\firstpageheader{ECS 198}{Homework 1}{April 7, 2016}
\runningheader{ECS 198}
              {Homework 1, Page \thepage\ of \numpages}
              {April 7, 2016}
\firstpagefooter{}{}{}
\runningfooter{}{}{}
\begin{tabular}{rl}
Name:  & \rule{0.25\linewidth}{\linethickness} \\ 
Date:  & \rule{0.25\linewidth}{\linethickness} \\ 
\end{tabular}
\vspace{5mm}

{\let\newpage\relax\maketitle}

\addpoints
\begin{center}
\gradetable[h][questions]
\end{center}

\begin{center}
	\fbox{\fbox{\parbox{5.5in}{\centering
	Answer the questions in the spaces provided.
	If you run out of room for an answer, continue on the back of the page
	}}}
\end{center}


\begin{questions}

  \question[5] 
  What is the purpose of Diffie-Hellman?
  \fillwithlines{.75in}
	\vspace{\stretch{1}}

  \question[5]
  What is the difference between Symmetric and Asymmetric Encryption?
  \fillwithlines{.75in}
	\vspace{\stretch{1}}

  \question[15]
  What are the two properties of Digital Signatures and why are they important?
  \fillwithlines{1in}
	\vspace{\stretch{1}}


  \clearpage


  \question[15]
  What are the three properties of Cryptographic Hash Functions and why are they important?
  \fillwithlines{1in}
  \vspace{\stretch{1}}

  \question[10]
  What is the point of message digests?
  \fillwithlines{.5in}
  \vspace{\stretch{1}}

  \question[2]
  What is a Merkle-Damgard transform?
  \begin{choices}
    \choice A type of tree in which hashes are combined to make a root that can be used to verify
    a hash's existence.
    \choice A method of transforming variable length inputs to fixed length outputs
    \choice A method of exchanging private keys publically
    \choice A bitstring, determined by implementation, that the first block gets hashed with in SHA-256
  \end{choices}

  \question[2]
  What is the pigeonhole principle?
  \begin{choices}
    \choice A method of brute force searching for Cryptographic Hash collisions
    \choice A method of verifying a hash output given a key and message
    \choice The state of having spread-out outputs for a Cryptographic Hash Function
    \choice The idea that if the input sample space is larger than output sample space, collisions must exist
  \end{choices}

  \question[2] The \fillin[Birthday Paradox] is a phenomenon in which the probability of collisions rises much faster than expected.
    
  \question[2] A \fillin[Search Puzzle] is a math problem that requires searching a large amount to find a solution without shortcuts

  \question[25] I am performing a Diffie Hellman Key Exchange with you. Given prime numbers 3 and 5, and secret numbers 18 and 23, 
  what is our shared key?
  \vspace{\stretch{5}}



  \clearpage

	\question Is it true that \(x^n + y^n = z^n\) if \(x,y,z\) and \(n\) are
	positive integers?. Explain.

	\vspace{\stretch{1}}
	\question Is it true that \(x^n + y^n = z^n\) if \(x,y,z\) and \(n\) are positive integers?. Explain.
	\vspace{\stretch{1}}

	\question Prove that the real part of all non-trivial zeros of the function \(\zeta(z)\) is \(\frac{1}{2}\)
	\vspace{\stretch{1}}
	\clearpage

	\question Compute \[\int_{0}^{\infty} \frac{\sin(x)}{x}\]
	\vspace{\stretch{1}}
	\question Which of these guys invented time

	\begin{oneparchoices}
		\choice Stephen Hawking 
		\choice Albert Einstein
		\choice Isaac Newton
		\choice This makes no sense
	\end{oneparchoices}

	\question Which of these guys published a paper on Browninan Motion

	\begin{checkboxes}
		\choice Stephen Hawking 
		\choice Albert Einstein
		\choice Isaac Newton
		\choice I don't know
	\end{checkboxes} 

	\clearpage

	\question Given the equation \(x^n + y^n = z^n\) for \(x,y,z\) and \(n\) positive
	integers. 
	\begin{parts}
		\part[10] For what values of $n$ is the statement in the previous question true?
		\vspace{\stretch{1}}

		\part[10] For $n=2$ there's a theorem with a special name. What's that name?
		\vspace{\stretch{1}}


		\part[10] What famous mathematician had an elegant proof for this theorem but there was
		not enough space in the margin to write it down?
		\vspace{\stretch{1}}

	\end{parts}

	\question[20] Compute \[\int_{0}^{\infty} \frac{\sin(x)}{x}\]

	\vspace{\stretch{1}}
\end{questions}


\end{document}
