\title{\bf Lecture 4 - Mining Incentives, Challenges and Future Directions\\}
\author{\bf Rylan Schaeffer and Vincent Yang\\}
\date{\bf \today \\}

\documentclass{article}
\renewcommand{\thesubsection}{\thesection.\alph{subsection}}
\usepackage{enumerate}
\usepackage{amsmath}
\usepackage{graphicx}
\setlength{\oddsidemargin}{0in}
\setlength{\evensidemargin}{0in}
\setlength{\textheight}{9in}
\setlength{\textwidth}{6.5in}
\setlength{\topmargin}{-0.5in}

\begin{document}
\maketitle

Note: This lecture is based on Princeton University's BTC-Tech: Bitcoin and Cryptocurrency Technologies Spring 2015 course.

\section*{Review}
\begin{itemize}
\item Need method to achieve distributed consensus
\item One option is proof-of-work, in which we balance two opposing goals:
\subitem challenging to validate transactions, to prevent interference
\subitem rewarding to validate transactions, to encourage people to participate
\item Accomplished through hash puzzles i.e. calculate nonce $r$ such that $H(previous hash | tx_1 | tx_2 | ... | tx_n | tx_s | r)$ has target number of leading zeroes
\item Two incentives:
\subitem Block reward i.e. $tx_s$ is a new coin
\subitem Transaction fees
\end{itemize}

\section*{To Be A Miner}
\begin{enumerate}
\item Listen for transactions
\item Maintain block chain and listen for new blocks
\item Assemble new block
\item Mine that block i.e. find the nonce $r$
\item Hope your block is accepted
\item Profit
\end{enumerate}

Note: Not all miners are working on the same problem! Miners (almost) always have unique set of transactions. Even if the transactions are identical, they're trying to send the transaction rewards and new block to themselves.

Protocol dictates that, on average, mining a new block should take about 10 minutes, adjusted about every two weeks.

\section{Mining Hardware}
\begin{itemize}
\item SHA-256 requires 32-bit words, 32-bit modular addition, some bitwise logic
\item Central Processing Unit mining works, but is slow
\item Graphics Processing Unit mining is faster, but wastes floating point unit and consumes more power.
\item Field-Programmable Gate Arrays mining is comparable to GPUs, better with bitwise logic, can be easily packed together, can be controlled from one centralized unit. Difficult to optimize 32-bit step addition
\item Application-Specific Integrated Circuits: best of all possible worlds
\end{itemize}

\section{Mining Costs}
\begin{itemize}
\item Technology quickly becomes antiquated
\item Cheaper in cold locations with cheap electricity
\item Shifted to professional companies
\item Model as Poisson process
\item Reduced variance when mining together
\end{itemize}

\section{Mining as a Group}
\begin{itemize}
\item Track each person's computational contribution, using "near misses" as a proxy measurement
\item Model 1: Pay per share. Pool manager pays flat fee per near miss above a set difficulty. No incentive to send valid blocks. High risk for pool manager.
\item Model 2: Pay proportionally. Low risk for pool manager.
\item Model 3: Pay someone to run ASIC cluster. Receive share of reward when new block is mined
\end{itemize}

\section{Pros and Cons of Proof-of-Work}
\begin{itemize}
\item Pros: Allow individuals and groups to make money, where otherwise impossible
\item Cons: Few possible attacks; leads to centralization; wasteful!
\end{itemize}

\section{Attacks on Proof-Of-Work}
\begin{itemize}
\item Forking attack: Double spending possible if thief controls $>$51\% of the network. Spends money, seller accepts because chain is long enough, then thief extends alternative chain, effectively erasing the previous transaction. May not require 51\%
\item Block-withholding attack: Solve a block, wait til someone else solves a block, then release yours to prevent their from being accepted. Increases your effective share of rewards.
\item Blacklisting: Can refuse to process transactions from certain public keys
\end{itemize}

\section{Disincentivizing Pool Formation}
\begin{itemize}
\item Note that pool formation requires two components:
\subitem Easy for miner to prove amount of work they are doing
\subitem Easy for miner to prove that they're following rules
\item Possible Solution: Change puzzles from Hash(transactions) to Hash(signature of transactions), where the secret key used to sign is the secret key corresponding to the recipient of the newly mined block
\subitem Pool operators either give out private key for pool members to use
\subitem or compute signatures themselves (10x more computationally expensive than hashing)
\item Does nothing to counter ASICs
\end{itemize}

\section{Countering ASICs}
\begin{itemize}
\item Alternative puzzle requirements:
\subitem Easy to verify solutions
\subitem Adjustable difficulty to control rate of mining
\subitem ``Program freeness": chances of winning should be proportional to computational power. Also known as memoryless process.
\item Trust-based networks e.g. Ripple
\item Memory-Hard Puzzles
\item Moving Target i.e. change puzzle to prevent pooling
\item Virtual Mining i.e. Proof-Of-Stake
\end{itemize}

\section{Trust-Based Networks}

\section{Memory-Hard Puzzles}
\begin{itemize}
\item Goal: Decentivize consolidation by making individual computers competitive with ASICs.
\item Method: memory-hard puzzles i.e. puzzles that require large amount of memory to solve or memory-bound puzzles i.e. puzzles that are slowed down by memory access time.
\item Also known as space-hard puzzles
\item Example: Scrypt. Used in Lytecoin. Two steps. 
\subitem 1) fill large buffer of RAM with pseudorandom data i.e. for i = 1 to N: V[i] = SHA-256(V[i-1])
\subitem 2) read and update the memory in pseudorandom order i.e. X = SHA-256(V[N-1]). for i = 1 to N, {j = X \% N; X = SHA-256(X \string^ V[j])}
\subitem Memory-hard because if V isn't stored in memory, recomputing X = SHA-256(X \string^ V[j]) takes $O(n^2)$ instead of $O(n)$ as $j$ is generated pseudorandomly.
\subitem Unfortunately, requires as much money to verify as does to compute
\subitem Eventually Litecoin revealed that Scrypt ASICs were better than ordinary hashing ASICs
\item Example: Balloon Hashing. New family of space hard password hashing functions recently invented at Stanford by Boneh and Corrigan-Gibbs
\item Example: Cuckoo Cycle. Based on difficulty of finding cycles in a graph generated from a cuckoo hash table.
\end{itemize}

\section{Moving Target Puzzles}
\begin{itemize}
\item Strategy: change puzzles occasionally so that ASICs are less cost-competitive
\end{itemize}

\section{Virtual Mining}
\begin{itemize}
\item Want alternative basis for "voting power" in distributed consensus protocol
\item Proof-of-Work also requires consuming real world goods (energy, equipment) to create virtual dollars.
\item Idea: allocate mining power to currency holders in proportion to how much currency they hold
\subitem: Advantages: removes centralization because ASICs have less of an advantage, currency holders have stake in future health of ecosystem
\subitem: Disadvantages: Richest participants given easiest mining puzzle; vulnerable to burst attacks (i.e. store up stake)
\item Example: Peercoin
\subitem Uses combination of Proof-of-Work and Proof-Of-Stake
\subitem To mine, solve SHA-256, but difficulty based on age of coin that is "consumed" by validating that block
\item Nothing-at-Stake Problem (i.e. Stake-Grinding): attacker A, with less than 50\% of stake, tries to create fork of k blocks
\subitem If fork succeeds, success. If fork fails, no cost
\item Someone who controls $>$51\% of currency can exclude others and mine only their own blocks
\end{itemize}

\section{Proof-of-Useful-Work}
\begin{itemize}
\item Idea: Use computing energy for some benefit to society instead of wasting it. Examples include:
\item Example: Distributed computing projects e.g. SETI at Home, Folding at Home (protein structure folding)
\subitem Problem: solution space was not equiprobable
\item Example: Great Internet Mersenne Prime Search
\subitem Problem: solutions too rare
\item Primecoin uses proof-of-useful work, by requiring miners to find Cunningham chains of prime numbers. Questionable how useful this is.
\end{itemize}


\section{Proof-of-Storage}
\begin{itemize}
\item Also called Proof-of-Retrievability
\item Idea: use miners' storage power to help with some archival project/distributed computing problem
\item Used in Permacoin as follows:
\subitem Let $F$ be a large file that we want stored e.g. Large Hadron Collider backup data of several hundred petabytes
\subitem Construct Merkle Tree of $F$
\subitem Each miner given random subset of F called $F_M \subset F$
\subitem Go read the paper if you care.
\end{itemize}


\end{document}