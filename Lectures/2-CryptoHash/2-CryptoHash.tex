\title{\bf Lecture 1 - Cryptographic Hash Functions\\}
\author{\bf Rylan Schaeffer and Vincent Yang\\}
\date{\bf \today \\}

\documentclass{article}
\renewcommand{\thesubsection}{\thesection.\alph{subsection}}
\usepackage{enumerate}
\usepackage{amsmath}
\usepackage{graphicx}
\setlength{\oddsidemargin}{0in}
\setlength{\evensidemargin}{0in}
\setlength{\textheight}{9in}
\setlength{\textwidth}{6.5in}
\setlength{\topmargin}{-0.5in}

\begin{document}
\maketitle

Note: This lecture is based on Princeton University's BTC-Tech: Bitcoin and Cryptocurrency Technologies Spring 2015 course.

\section*{Terms}
\begin{itemize}
\item Set : group of objects represented as a unit
\item Alphabet : finite, non-empty set
\item String : finite sequence of characters from common alphabet, including empty string $\varepsilon$
\item Language : set of strings over common alphabet
\end{itemize}

\section*{Hash Functions}
\begin{itemize}
\item $H: \{0,1\}^* \rightarrow \{0,1\}^k$, for fixed k e.g. 256
\item Should be efficiently computable O(n)
\item Example: mod operator
\end{itemize}

\section*{Cryptographic Hash Function}
Two additional properties
\begin{itemize}
\item Collision Resistant: Computationally infeasible to find $x, y$ such that $x \neq y$ and $H(x) = H(y)$
\subitem mod operator is not collision resistant
\subitem collisions exist by pigeonhole principle - hence, computationally infeasible
\subitem birthday paradox reduces difficulty of finding collisions
\subitem can also call "binding," since once hash is published, you cannot replace input value with another input value without modifying the hash output

\item Hiding: Computationally infeasible to find $x$ given $H_{given}$ such that $H(x) = H_{given}$
\subitem Frequently, cryptographic hash functions will be called one-way hash functions
\subitem Frequently, message space is too small. Append nonce (i.e. random value) r to grow message space such that computationally infeasible to find $x$ such that $H(x|r) = H_{given}$

\end{itemize}

\section*{Applications}
\begin{itemize}
\item Message Digest
\subitem Create summary (or ``digest") of block of text
\subitem Suppose I have $msg$ and $H$ is a cryptographic hash function. Then I know that $H(msg)$ or perhaps $H(msg|r)$ (where r is a random value and is needed because the message is predictable), will produce a hash value that no other block of text will.
\subitem Example: cryptographic checksums
\item Commitments
\subitem Analogous to sealed envelope on the table
\subitem Hiding ensures no one can "reverse engineer" the contents. Collision-resistant guarantees to the other party that you are bound to the value you initially put in.
\end{itemize}

\section*{Puzzle Friendliness}
\begin{itemize}
\item Search Puzzle
\subitem Given $H$, target set $Y$, and value $x$
\subitem Goal: find r such that $H(x|r) \in Y$
\item Puzzle friendly if no solving strategy for puzzle other than trying random guesses at $r$
\item Examples: $0|\{0,1\}^{k-1}$, $00|\{0,1\}^{k-1}$, $000|\{0,1\}^{k-1}$
\subitem P(l leading zeroes) = $\frac{1}{2^l}$, can use geometric distribution's cumulative distribution function to model likelihood of observing a ``hit" after a given number of failures
\item Useful for mining, which we will get to later
\end{itemize}

\section*{Hash Structures}
\begin{itemize}
\item Hash pointer : hash of data. Gives way to verify information hasn't changed, much like pointer gives a way to retrieve location of information
\item Hash linked list (block chain) : Each block has hash of previous block plus new data. Head is hash of most recent block.
\subitem Tamper-evident log
\item Hash tree (Merkle Tree) : binary tree of data blocks. Proof of membership and proof of non-membership in $log(n)$, so faster than hash linked list. Can also sort.
\item Can combine. Block chain is usually hash linked list of hash trees  (draw picture on board)
\end{itemize}
\end{document}