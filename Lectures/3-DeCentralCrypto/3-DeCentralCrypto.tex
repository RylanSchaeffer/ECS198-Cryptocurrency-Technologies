\title{\bf Lecture 3 - Centralized and Decentralized Cryptocurrencies\\}
\author{\bf Rylan Schaeffer and Vincent Yang\\}
\date{\bf \today \\}

\documentclass{article}
\renewcommand{\thesubsection}{\thesection.\alph{subsection}}
\usepackage{enumerate}
\usepackage{hyperref}
\usepackage{amsmath}
\usepackage{graphicx}
\setlength{\oddsidemargin}{0in}
\setlength{\evensidemargin}{0in}
\setlength{\textheight}{9in}
\setlength{\textwidth}{6.5in}
\setlength{\topmargin}{-0.5in}

\begin{document}
\maketitle

Note: This lecture is based on Princeton University's BTC-Tech: Bitcoin and Cryptocurrency Technologies Spring 2015 course.


\section*{Centralization vs. Decentralization}
\section*{Centralized Banking}
\begin{itemize}
  \item Centralize: to concentrate under a single authority
  \item Centralized banking means there is a single institution that manages supply, inflation, and interest.
  \item Cryptocurrency has to satisfy:
    \subitem Mathematically complex (to avoid fraud and hacker attacks)
    \subitem Decentralized but with \emph{adequate consumer safeguards and protection}
    \subitem Preserve user anonymity without being a conduit for tax evasion, money laundering, etc.
\end{itemize}
\section*{Advantages to Centralization}
\begin{itemize}
  \item Automation:
    \subitem Easily manage a large number of keys e.g. Mastercard Europe
    \subitem Maintain secure infrastructure and improve operations/efficiency
  \item Centralized Monitoring:
    \subitem Record everything that happens easily; brings transparency
  \item Centralized Policy
  \item Easily update and track keys
  \item Easily update cryptographic schemes - swap out algorithms
\end{itemize}

\section*{CentralizedCoin}
\begin{itemize}
  \item I can generate coins, and give them unique ID's. I also sign these coins.
  \item I can pass them to anyone else - I sign the transaction; recipient can prove it's valid because it has my signature.\\
    Recipient can sign to pass to someone else. 
  \item Chain of hash pointers can be used to follow it back. = verify
  \item Double Spending Problem
  \item Only I can write on the chain - everything has to pass through me
  \item This is centralized; how do you trust me? 
\end{itemize}

\section*{Centralized Cryptocurrencies}
\begin{itemize}
  \item E-Gold (1996)
    \subitem Operated by Gold and Silver Reserve inc 
    \subitem Let users open an account denominated in gold; could make instant transfers
    \subitem Grew to 5 million accounts; processing over 2 billion a year
    \subitem "e-Gold Special Purpose Trust" - actually held the gold; could see gold bars with serial numbers per acct.
    \subitem Hackers used flaws in Microsoft Windows OS's and phishing to compromise millions of e-gold accounts
    \subitem People thought it was \emph{anonymous}, but really it was \emph{pseudonymous}. Law enforcement identified many.
    \subitem Ponzi schemes via. eBay 
    \subitem Patriot Act, after Sept 11, made operating a money transmitter business without a state money transmitter license a federal crime.
    \subitem Taken down 2007-2013; inability to provide reliable user identification and cut off illegal activity
    \subitem PayPal has done a better job, but still has to deal with the same problems.
    \subitem KYC - process of verifying clients' identity
  \item Liberty Reserve
    \subitem Shut down, also by Patriot Act, in May 2013. 
  \item E-Gold and Liberty Reserve were popularly used for money laundering and shut down
  \item Can be shut down by the government at any time
  \item DigiCash by Chaum 1990
    \subitem Store money as data on your computer
    \subitem Transfer anonymously
    \subitem Lacked decentralization; the company's servers were used
    \subitem Went bankrupt in 1998
\end{itemize}

\section*{Decentralized Cryptocurrencies}
\begin{itemize}
  \item Decentralization is not all or nothing
    \subitem Partially decentralized - SMTP (email)
  \item Bitcoin and Decentralization
    \subitem How does Bitcoin deal with decentralization?
    \begin{enumerate}
      \item Who maintains ledger of transactions?
      \item Who determines which transactions are valid/invalid?
      \item Who creates new coins
      \item Who chooses when rules change
      \item How do bitcoins gain value
    \end{enumerate}
    \subitem Concensus (distributed concensus)
    \subitem \emph{Distributed concensus protocol}: two properties
    \begin{enumerate}
      \item Must end with all nodes in agreement, and value has to have been generated by honest node
      \item When someone wants to make a transaction, the person broadcasts to the nodes that make up the network. 
      \subitem \emph{There is no requirement for the recipient to be on the network}
    \end{enumerate}
    \subitem Must come to concensus on which transactions were broadcast in what order
    \subitem Each node has:
    \begin{enumerate}
      \item Single, global ledger that each node has a copy of 
      \item Pool of transactions that have been received but not verified (varies from node to node)
    \end{enumerate}
  \item How do nodes come to consensus? 
    \subitem At regular intervals, every node proposes its own pool to be next block
    \subitem Consensus protocol with each node's input as its own block
    \subitem If this protocol succeeds, then a valid block will be chosen - it doesn't matter how many people propose this block
    \subitem Doesn't matter if transactions get left out; they could just be in the next block
  \item Problems:
    \subitem Latency, not all nodes connected, internet connection, malicious nodes
    \subitem Global time does not exist
  \item Byzantine Generals
  \item Paxos
    \subitem Makes compromises - never produces inconsistent result, but under rare conditions, protocol can get stuck
  \item How Bitcoin breaks traditional assumptions
    \subitem Works better in practice than in theory - no accurate model yet exists
    \subitem Only solves problems in currency context due to incentives(not distributed databases, which is where the problem originated)
    \subitem Embraces randomness - concensus happens over an hour, nodes can't be certain of what's in/out; the odds just change exponentially
    
  \item Block Chain
    \begin{itemize}
      \item Consensus without identity
      \item Sybil attack
        \subitem Can't gain more power by having more accounts
        \subitem Satoshi's original paper had 1 cpu = 1 vote
      \item Implicit Concensus:
        \subitem Chosen node chooses what the next block is; voting is by what is extended by the others
      \item Bitcoin consensus algorithm (simplified)
        \begin{enumerate}
          \item New transactions broadcast to all nodes
          \item Each node collects transactions
          \item Random node gets to broadcast its block per round
          \item Other nodes accept only if valid
          \item Nodes show acceptance through including block in hash for next block
        \end{enumerate}
      \item Attacks
        \subitem Stealing - even if Alice gets to decide next block, she can't steal because she has to create valid transaction; can't forge signatures
        \subitem Denial of Service - even if Alice never validates Bob's transactions, an honest node will eventually do so
        \subitem Double Spend - Say Alice pays Bob, and an honest node broadcasts this. and Bob accepts that he's been paid. Alice then gets to broadcast her own transaction.\\
        She then makes a block with the emph{prevBlock} hash as the one before her payment to Bob. Only one of these blocks will be accepted.
      \item Blocks have a tendency to extend the block they hear about first
      \item Orphan Block
      \item Zero-confirmation transaction
        \subitem Bob gives Alice product before transaction has been verified
        \subitem 6 blocks; double spend probability goes down exponentially
        \subitem Never a 100\% guarantee
    \end{itemize}

  \item Incentives/Proof of work
    \subitem Motivation to subvert the process (picking a hopefully honest node), so reward honest nodes
    \subitem HashCash - with SHA-256
    \subitem Can't penalize those who try to double spend, since there's no way to tell
    \subitem Use bitcoins to incentivize honest nodes - mining. Reward only if it becomes legitimate transaction
    \subitem Every 210,000 blocks (4 years), block reward is cut in half. Geometric sum - 21 million bitcoins
  \item Incentives Part 2 - Transaction fees
    \subitem Incentive to have your transaction verified
  \item New Problems With Incentives
    \subitem Random node
    \subitem Everyone wants to run nodes for rewards
    \subitem Sybil nodes to subvert process
    \subitem Solution: proof of work
  \item Proof of work
    \subitem Selecting nodes based on processing power/proportional
    \subitem Hopefully not monopolized
    \subitem proof of stake - proportion to ownership of currency (used in other cryptocurrencies)
    \subitem Hash puzzles - to make blocks, it needs to find a nonce where\\
    $ H(nonce\ ||\ prev\_hash\ ||\ tx\ ||\ tx\ ||\ ...\ ||\ tx)\ <\ target $
    \subitem Nonce: Random number that is only used once
    \subitem Hash puzzle properties: difficult, parameterizable cost (10 minutes variable target), trivial to verify
    \subitem 10 minutes: reduce inefficiency from having many blocks
    \subitem $ mean time to next block = \frac{10 minutes}{fraction\ of\ hash\ power} $
  \item The miner gains if reward > cost
    \subitem reward = block reward + tx fees
    \subitem mining cost = hardware cost + operating costs
\end{itemize}


\section*{Changing the rules}
\begin{itemize}
  \item Two types of changes - soft forks; hard forks
    \subitem Soft forks are forward compatible; new rules are subset of old rules. Only applied if over 51\% agree.
    \subitem Hard forks are backward compatible; old rules are subset of new rules. Everyone needs to upgrade to new.
\end{itemize}



Source: \url{http://people.dsv.su.se/~matei/courses/IK2001_SJE/Chaum90.pdf}\\
Source: \url{http://blog.koehntopp.de/uploads/Chaum.BlindSigForPayment.1982.PDF}

\end{document}
