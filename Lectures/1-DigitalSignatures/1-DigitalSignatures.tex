\title{\bf Lecture 1 - Digital Signatures\\}
\author{\bf Rylan Schaeffer and Vincent Yang\\}
\date{\bf \today \\}

\documentclass{article}
\renewcommand{\thesubsection}{\thesection.\alph{subsection}}
\usepackage{enumerate}
\PassOptionsToPackage{hyphens}{url}\usepackage{hyperref}
\usepackage{amsmath}
\usepackage{hyperref}
\usepackage{graphicx}
\setlength{\oddsidemargin}{0in}
\setlength{\evensidemargin}{0in}
\setlength{\textheight}{9in}
\setlength{\textwidth}{6.5in}
\setlength{\topmargin}{-0.5in}

\begin{document}
\maketitle

Note: This lecture is based on Princeton University's BTC-Tech: Bitcoin and Cryptocurrency Technologies Spring 2015 course.

\section*{Cryptography}
\begin{itemize}
  \item If you want to make information secret, hide its existence or make it intelligible
  \item Symmetric and Asymmetric Cryptography
\end{itemize}

\section*{Symmetric Cryptography}
\begin{itemize}
  \item An encryption scheme \emph{SE = (K, E, D)} in which the sender and receiver use the same key to encrypt and decrypt information.
    \subitem Randomized \emph{key generation} algorithm \emph{K} that returns a string \emph{k}. \emph{Keys(SE)}  is the set of possible bitstrings that \emph{K} can output (keys).
    \subitem \emph{Encryption Algorithm}  \emph{E} that takes a key \emph{k} $ \in $ \emph{K}.
    \subitem \emph{decryption} algorithm \emph{D} that inputs a key \emph{k} $ \in $ \emph{Keys(SE)} and
    \emph{plaintext} \emph{M} $ \in  \{0, 1\}* $.\\ 
    $ M \gets D\textsubscript{K}(C) $
  \item Message Space
    \subitem The set of possible messages
    \subitem Key Space
    \subitem Ciphertext Space/Cipher Space
  \item Caesar Cipher
    \subitem Rotate each letter n letters down the alphabet
    \subitem Key Space 26 - 1
  \item One Time Pad
  \item Diffie-Hellman Key Exchange
    \subitem A method of publically creating cryptographic keys.
    \subitem Trapdoor Function: A function that is easy to compute in one direction, but not the other.
    \subitem Perfect Forward Secrecy: The property in which compromised long-term keys do not invalidate the 
    integrity of a session key.
    \subitem modulus = remainder
    \begin{enumerate}
      \item Choose a modulus $p=23$ and base $g=5$. 
        \subitem \emph{g} is a Primitive root modulo. A number \emph{g} is a \emph{Primitive root modulo}
        if every number \emph{a} coprime to \emph{n} is congruent to a power \emph{g} of modulo emph{n}
        \subitem Restated: For every integer \emph{a} coprime to \emph{n}, there exists an integer \emph{k} such that
        $g\textsuperscript{k} \equiv a\ (mod\ n)$.
        \subitem Restated: \emph{g} is a generator for the multiplicative group of integers modulo \emph{n}.
      \item Alice chooses a secret integer $a = 6$, and sends Bob $A = g^{a}\ mod\ p$.
        \subitem $ A = 5\textsuperscript{6}\ mod\ 23 = 8$
      \item Bob chooses secret integer $b = 15$, and sends Alice $B = g^{b}\ mod\ p$.
        \subitem $B = 5^{15}\ mod\ 23 = 19 $
      \item Alice computes $ s = B\textsuperscript{a}\ mod\ p$.
        \subitem $s = 19\textsuperscript{6}\ mod\ 23 = 2$
      \item Bob computes $ s = A\textsuperscript{b}\ mod\ p$.
        \subitem $s = 8\textsuperscript{15}\ mod\ 23 = 2$
      \item They reached the same number because under mod p:
        \subitem $A\textsuperscript{b}\ mod\ p = g\textsuperscript{ab}\ mod\ p = g\textsuperscript{ba}\ mod\ p = B\textsuperscript{a}\ mod\ p$
        \subitem $= (g^{a}mod\ p)^{b}\ mod\ p = (g^{b}\ mod\ p)^{a}\ mod\ p$
    \end{enumerate}
\end{itemize}

\section*{Asymmetric Cryptography}
\begin{itemize}
  \item Properties
    \subitem Only you can create a signature, but anyone can verify its validity
    \subitem Tied to a document
    \subitem Ex: Actual mail
  \item Public/Private keys
  \item Digital signature scheme
    \subitem $ (sk, pk) := generateKeys(keySize) $ Randomized key generation
    \subitem $ sig := sign(sk, message) $ Encryption Algorithm
    \subitem $ isValid := verify(pk, message, sig) $ Deterministic Decryption Algorithm
  \item Ensure only one person can decrypt your message
  \item Ensure a message was created by someone
  \item Unforgeability Game
    \subitem Challenger has $ (sk,\ pk) $ and attacker has $ pk $.
    \subitem Attacker can send over $ m_0, m_1, m_2... m_n $ and get back $ sign(sk, m_0) $.
    \subitem Attacker has to send $ M,\ sig $ where $ M \notin {m_0, m_1, m_2, ...m_n} $.
    \subitem Challenger verifies: $ verify(pk, M, sig) $.
    \subitem Existential Forgery: Creation of a message/signature pair $(m, \sigma)$, where $\sigma$ isn't created by signer.
    \subitem Existentially Unforgeable: The chance of successfully forging a message is negligible to the point that
    it will never happen in practice.
  \item Other types of Forgery for signatures:
      \subitem Multiplication Forgery
      \subitem Selective Forgery
      \subitem Universal Forgery
  \item RSA
    \subitem Ron Rivest, Adi Shamir, Leonard Adleman
    \begin{itemize}
      \item Basic Principle: $ (m\textsuperscript{e})\textsuperscript{d}\ mod\ n = m $
        \subitem Even with \emph{m}, \emph{e}, and \emph{n}, it is extremely difficult to find \emph{d}.
      \item Key Distribution
        \subitem Distribute public key $ (n, e) $. 
      \item Encryption
        \subitem Change \emph{M} to integer \emph{m}. Make sure $ 0 \leq m < n $ and $ gcd(m, n) = 1 $ through an agreed padding scheme.
        \subitem Compute ciphertext \emph{c} through $ c \equiv (m\textsuperscript{e})\textsuperscript{d} \equiv m\ mod\ n $
      \item Decryption
        \subitem Use private key exponent \emph{d} with $ c\textsuperscript{d} \equiv (m\textsuperscript{e})\textsuperscript{d} \equiv m\ mod\ n $
        \subitem Then, reverse emph{m} to \emph{M} using the padding scheme.

      \item Key Generation
        \subitem Choose different prime numbers \emph{p} and \emph{q}. Find $ n = pq $.
        \subitem \emph{n} is the key. $ len(n) = keyLength $.
        \subitem Find $ \phi(n) $.
        \begin{enumerate}
          \item Choose two large, different primes $ p $ and $ q $. Find $ n = p \cdot q $
            \subitem \emph{n} is the key. $ len(n) = key_length $.
          \item Find $ \phi(n) $. This turns out to be $ (p - 1) (q - 1) $
            \subitem $ \phi $ is Euler's totient function. $ \phi(n) $ is the number of
            positive integers less than $ n $ that are coprime to $ n $. $ \phi(1) = 1$.
            \subitem Two numbers are \emph{Coprime} when the only positive integer that divides them is $ 1 $. 
            \subitem Given $ n $, a prime number, $ \phi(n) = n-1 $. e.g. $ n = 5 $, so $ 1, 2, 3, 4 $ are coprime to $ 5 $. 
            \begin{enumerate}
              \item However, for composite numbers, it works for some but not others. 
              \item e.g. $ 15 = 3 \cdot 5 $ and $ \phi(15) = \phi(3) \cdot \phi(5) = 2 \cdot 4 = 8 $. 
              \item But doesn't hold for 4, 8, 9. 
            \end{enumerate}
            \subitem If \emph{m} and \emph{n} are coprime, $ \phi(m \cdot n) = \phi(m) \cdot \phi(n) $.
          \item Find an integer \emph{e} where $ 1 < e < \phi(n) $ and $ gcd(e, \phi(n)) = 1 $. This means \emph{e} and $ \phi{n} $ are coprime.
          \item Find \emph{d}, the modular multiplicative inverse of $ e(mod\ \phi(n)) $
            \subitem This means: Solve for \emph{d} where $ d \cdot e \equiv 1\ (mod\ \phi(n)) $
            \subitem \emph{e} should have short length, and is usually $ 2\textsuperscript{16} + 1 = 65,537 $.
            \subitem \emph{e} is the public key, along with \emph{n}.
            \subitem \emph{d} is the private key.
        \end{enumerate}
      \item Example:
        \begin{enumerate}
          \item Find two distinct prime numbers
            \subitem $ p = 11 $ and $ q = 7 $
          \item Find $ n = pq $
            \subitem $ n = 11 \cdot 7 = 77 $
          \item Compute the totient of n
            \subitem $ \phi(77) = (11 - 1)(7 - 1) = 60 $
          \item Choose \emph{e} such that $ 1 < e < 60 $, where \emph{e} is coprime to \emph{60}. 
            \subitem Let $ e = 17 $; check that $ 60 $ is not divisible by $ 17 $.
          \item Compute \emph{d}. Process is below.
            \subitem $ d = 53 $
            \subitem $ d \cdot e\ mod\ \phi(n) = 1 $
            \subitem $ 53 \cdot 17\ mod\ 60 = 1 $
          \item The public key is $ n = 77 $ and $ e = 17 $. 
            \subitem $ c(m) = m\textsuperscript{17}\ mod\ 77 $
          \item The private key is $ d = 53 $.
            \subitem $ m(c) = c\textsuperscript{53}\ mod\ 77 $
          \item To encrypt $ m = 65 $, 
            \subitem $ c = 65\textsuperscript{17}\ mod\ 77 = 32 $
          \item To decrypt $ c = 32 $, 
            \subitem $ m = 32\textsuperscript{53}\ mod\ 77\ = 65 $. 
        \end{enumerate}
      \item Calculating d for above: use Extended Euclid's Algorithm 
        \subitem Basically finding $ gcd $ with Euclid's Algorithm, but reversed.
        \begin{align*}
          \phi(77) &= 60 \\
          e \cdot d\ mod\ 60 &= 1 \\
          17 \cdot d\ mod\ 60 &= 1 \\
          60 &= 3(17) + 9 \\
          17 &= 1(9) + 8 \\
          9 &= 1(8) + 1 \\
        \end{align*}
        \subitem Once we find 1, rewrite as:
        \begin{align*}
          1 &= 9 - 1(8) \\
          8 &= 2(9) - 17 \\
          9 &= 60 - 3(17) \\
        \end{align*}
        \subitem Therefore,
        \begin{align*}
          1 &= 9 - (17 - 1(9)) \\
          1 &= 2(9) - 17 \\
          1 &= 2(60 - 3(17)) - 17 \\
          1 &= 2(60) - 7(17) \\
          \phi(77) - 7 &= d \\
          d &= 53 \\
      \end{align*}
      \item Calculating d for above: use Extended Euclid's Algorithm 
        \subitem Basically finding \emph{gcd} with Euclid's Algorithm, but reversed.
      \item Fermat's little theorem states $ a\textsuperscript{p} = a\ mod\ p $. This is equal to
        $ a\textsuperscript{p-1}=1\ mod\ p $. 
      \item For RSA, this is insufficient. You need the Euler-Fermat generalisation:
        \subitem $a^{\phi(n)} = 1$
        \subitem Group Theory: Multiplication for some sets of integers makes a group under modulo, if all the elements are coprime to the modulo used.
        \subitem E is coprime to $\phi(pq)$. Group theory says there exists some integer that acts uniquely as an inverse and transforms
        under multiplication to the identity.
        \subitem The identity element for multiplication is $1$. The inverse is $d$.
    \end{itemize}
  \end{itemize}
  Source: \url{http://crypto.stackexchange.com/questions/388/what-is-the-relation-between-rsa-fermats-little-theorem?lq=1}
\end{document}
