\title{\bf Lecture 3 - Centralized and Decentralized Cryptocurrencies\\}
\author{\bf Rylan Schaeffer and Vincent Yang\\}
\date{\bf \today \\}

\documentclass{article}
\renewcommand{\thesubsection}{\thesection.\alph{subsection}}
\usepackage{enumerate}
\usepackage{hyperref}
\usepackage{amsmath}
\usepackage{graphicx}
\setlength{\oddsidemargin}{0in}
\setlength{\evensidemargin}{0in}
\setlength{\textheight}{9in}
\setlength{\textwidth}{6.5in}
\setlength{\topmargin}{-0.5in}

\begin{document}
\maketitle

Note: This lecture is based on Princeton University's BTC-Tech: Bitcoin and Cryptocurrency Technologies Spring 2015 course.

\section*{Agenda}
\begin{itemize}
  \item Final Project
    \subitem Rudimentary cryptocurrency
    \subitem Groups vs. Pairs vs. Single
    \subitem Report on some study/article about Bitcoin/Cryptocurrencies
    \subitem Other ideas
  \item Homework 3
  \item Why decentralize?
  \item Problems to solve/How proof of work solves them
  \item Mining/Motivations/Why it's important (Proof of work)
  \item Shortest vs. Longest blockchain
  \item Double spending
  \item 51\% attacker
\end{itemize}

\section*{Advantages to Centralization}
\begin{itemize}
  \item Automation:
    \subitem Easily manage a large number of keys e.g. Mastercard Europe
    \subitem Maintain secure infrastructure and improve operations/efficiency
  \item Centralized Monitoring:
    \subitem Record everything that happens easily; brings transparency
  \item Centralized Policy
  \item Easily update and track keys
  \item Easily update cryptographic schemes - swap out algorithms
\end{itemize}

\section*{CentralizedCoin}

\section*{Decentralized Cryptocurrencies}
\begin{itemize}
  \item Decentralization is not all or nothing
    \subitem Partially decentralized - SMTP (email)
  \item How does Bitcoin deal with Decentralization?
    \subitem Problems to address, since it's no longer centralized
    \begin{enumerate}
      \item Who maintains ledger of transactions?
      \item Who determines which transactions are valid/invalid?
      \item Who creates new coins
      \item Who chooses when rules change
    \end{enumerate}
\end{itemize}

\section*{Distributed Consensus (Solves 1 and 2)}
\begin{itemize}
  \item Paxos - Method for consensus
  \item All good nodes agree on the same value (proposed by a good node)
    \subitem A good node is one that is being honest
  \item How it works:
    \begin{itemize}
      \item All nodes have a sequence of blocks of transactions they have reached consensus on; order is important
      \item Each node has a set of outstanding transactions - this solves the problem where not all nodes are aware of all transactions.
      \item Each transaction is broadcast to all
      \item Doesn't matter if transactions are left out - they get included in the next block.
      \item A random node gets to broadcast its block per round
      \item Other nodes accept only if valid, and show through including block in hash for next block.
      \item Paxos is a Two Phase Commit (2PC): Coordinator suggests value to all nodes, Coordinator (on receiving enough
          yes's), says value is final - Update. (Problem of falling coordinator is solve by everyone being able).\\
        \subitem Two phases are called \emph{Voting Phase} and \emph{Commit Phase}.
    \end{itemize}
  \item Consensus without identity - Block Chain
    \begin{itemize}
      \item Solution: Proof of Work (explanation next)
      \item Bitcoin nodes don't have identity; pseudonymity vs. anonymity
      \item Implicit Consensus: Random node picks next block in the chain, and other nodes vote by extending or ignoring.
        \subitem Implicit because implied though not plainly expressed
      \item The more confirmations, the better the choice. Bitcoin uses 6.
    \end{itemize}
\end{itemize}
\section*{The Solution: Proof of work:}
\begin{itemize}
  \item Select nodes in proportion to computational power
  \item \emph{Give example of leading 0's}
  \item Difficult to compute
  \item Puzzle Friendliness: No solving strategy for finding $H(k | x)=y$ is better than trying random values of x.
  \item Motivation to subvert the process (picking a hopefully honest node), so reward honest nodes
  \item Use bitcoins to incentivize honest nodes - mining. Reward only if it becomes legitimate transaction
  \item Every 210,000 blocks (4 years), block reward is cut in half. Geometric sum - 21 million bitcoins
  \item Nonce published as part of the block.
  \item Proof of work details
    \begin{itemize}
      \item Mining: Hash function with nonce appended to beginning (ex: 0's)
      \item HashCash - with SHA-256 (used twice)
      \item Less than some value
      \item Items used: version, prevBlockHash, merkleRoot, timestamp, bits, nonce
      \item Selecting nodes based on processing power/proportional
      \item Hash puzzles - to make blocks, it needs to find a nonce where
      \item $ H(nonce\ ||\ prev\_hash\ ||\ tx\ ||\ tx\ ||\ ...\ ||\ tx)\ <\ target $
      \item Nonce: Random number that is only used once
      \item Hash puzzle properties: difficult, parameterizable cost (10 minutes variable target), trivial to verify
      \item 10 minutes: reduce inefficiency from having many blocks
      \item $ mean time to next block = \frac{10 minutes}{fraction\ of\ hash\ power} $
      \item proof of stake - proportion to ownership of currency (used in other cryptocurrencies)
      \item http://www.righto.com/2014/09/mining-bitcoin-with-pencil-and-paper.html
    \end{itemize}
  \item Incentives: Solves 3
    \begin{itemize}
      \item Can't penalize, but can reward nodes for working correctly.
      \item \emph{Incentive 1}: Block Reward (25 BTC, halves every 4 years). (Solves 3)
        \subitem 21 million max - block reward is how new coins are created; run out in 2140.
      \item \emph{Incentive 2}: Transaction Fee
        \subitem Incentive to have your transaction verified
      \item Remaining problems: 
        \subitem How to pick random node
        \subitem How to avoid free-for-all rewards
    \end{itemize}
  \item The miner gains if reward $>$ cost
    \subitem reward = block reward + tx fees
    \subitem mining cost = hardware cost + operating costs
  \item Aspects of decentralization in Bitcoin
    \begin{itemize}
      \item Peer to peer network, open to anyone, low barrier for entry
      \item Mining is open to everyone
      \item Updates to software - core developers trusted by the community who have a lot of power (Solve Problem 4)
    \end{itemize}
  \item New Problems from being Decentralized and Attacks:
    \begin{itemize}
      \item Fischer-Lynch-Paterson: concensus impossible with a single faulty node.\\
        solved with 6 blocks 
        \subitem Can't penalize those who try to double spend, since there's no way to tell
      \item Blocks have a tendency to extend the block they hear about first
      \item Orphan Block
      \item Latency, not all nodes connected, internet connection, malicious nodes
      \item Attacks:
      \item Stealing - even if Alice gets to decide the next block, she can't steal because she has to create
        a valid transaction; can't forge signatures
      \item Denial of Service - even if Alice never validates Bob's transactions, an honest node will eventually do so.
      \item Double Spend - DRAW DIAGRAM\\ Say Alice pays Bob, and an honest node broadcasts this. Bob accepts that he's been paid. Alice then
        gets to broadcast her own transaction. She then makes a block with the \emph{prevBlock} hash as the one before her payment to Bob.
        Only one of these blocks will be accepted. 
      \item Sybil attack
        \subitem Can't gain more power by having more accounts
        \subitem Satoshi's original paper had 1 cpu = 1 vote
      \item Zero-confirmation transaction
        \subitem Bob gives Alice product before transaction has been verified
        \subitem 6 blocks; double spend probability goes down exponentially
        \subitem Never a 100\% guarantee
    \end{itemize}
  \item Solves the problem of not trusting a central authority - Problem 4
  \item 51\% attacker
    \begin{enumerate}
      \item CANNOT Steal Bitcoin
      \item CANNOT change block reward
      \item Suppress transactions
      \item Destroy confidence in Bitcoin
    \end{enumerate}
\end{itemize}


\section*{Changing the rules}
\begin{itemize}
  \item Two types of changes - soft forks; hard forks
    \subitem Soft forks are forward compatible; new rules are subset of old rules. Only applied if over 51\% agree.
    \subitem Hard forks are backward compatible; old rules are subset of new rules. Everyone needs to upgrade to new.
\end{itemize}



Source: \url{http://people.dsv.su.se/~matei/courses/IK2001_SJE/Chaum90.pdf}\\
Source: \url{http://blog.koehntopp.de/uploads/Chaum.BlindSigForPayment.1982.PDF}

\end{document}
